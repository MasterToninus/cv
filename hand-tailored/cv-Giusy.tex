%=======================================================================%
%      CV - non accademico
%=======================================================================%
%  --- NON GENERATO AUTOMATICAMENTE ---
%
%   non ha senso automatizzarlo per adesso, le cose da aggiungere sono poche per ora.
%
%
%
%=======================================================================%





\documentclass[a4paper]{article}


\usepackage[utf8]{inputenc}
\usepackage[T1]{fontenc}
\usepackage{fontawesome}
\usepackage[margin=1.5cm]{geometry}
% \addtolength{\topmargin}{-0.5cm}
\usepackage{xcolor}
\usepackage{hyperref}
\usepackage{fancyhdr}
\usepackage{lastpage}
\pagestyle{fancy}
\fancyhf{}
\rhead{Giuseppa Simioli}
\lhead{Curriculum Vitae}
\lfoot{\footnotesize\href{https://github.com/MasterToninus/cv}{Generated on \today} }
\rfoot{\thepage \hspace{1pt} / \pageref{LastPage}}

%\usepackage[backend=bibtex, style=nature]{biblatex}
%\addbibresource{../data/publications.bib}

\setlength\parindent{0pt}


%barcharts
\newcommand{\crule}[3][black]{\textcolor{#1}{\rule{#2}{#3}}}
\newcommand{\progbar}[4][black]{\crule[#1]{#3}{#4}\crule[#1!10!white!90]{#2}{#4}}

%titles
\newcommand{\block}[1]{\hrule \vspace{0.2cm} \textbf{\Large #1} \vspace{0.2cm}}
\newcommand{\Block}[1]{\hrule \vspace{0.2cm} \textbf{\huge #1} \vspace{0.2cm}}
\newcommand{\blockNp}[1]{\newpage \textbf{\Large #1} \vspace{0.2cm}}

%longvoice
%expected: Event_Title,Type,Url, Approx_Date, Organization, Location, Country
\newcommand{\longvoice}[8]{
    \begin{tabular}{p{0.83\linewidth} p{0.17\linewidth} }
        \textbf{\href{#3}{#2: #1}} & #4 
        \\ 
        \textit{#5, #6, #7;} & {\small\emph{#8}}
    \end{tabular}
    \vspace{.5em}
    }
%voice
%expected: what, when, where, url, info
\newcommand{\voice}[5]{\href{#4}{\textbf{#1}} \hfill #2 \\ \textit{#3} \\ {\small #5} \vspace{0.2cm} \\}
\newcommand{\skill}[2]{\textbf{#1} \hfill #2 \\}
\newcommand{\skillbar}[4]{\textbf{#1} \hfill \progbar{#2}{#3}{#4} \\}
\newcommand{\txt}[1]{#1\\}




%---------- Document -----------
\begin{document}


%Header
\begin{minipage}[c]{0.6\columnwidth}
    \center{\huge \textbf{Giuseppa SIMIOLI}}
    \\
    \flushleft{\normalsize \textit{\color{orange}Postdoctoral researcher in pure mathematics. Interested in functional programming, quantum algorithms, high performance computing, scientific programming and numerical simulations.}}
\end{minipage} 
\hfill
\begin{minipage}[t]{0.375\columnwidth}
    \vspace{-1.5em}
    \skill{\faPhone~ Phone (ita)}{+39~}
    \skill{\faPhone~ Phone (fra)}{+32~}
    \skill{\faEnvelope~ Mail}{\href{mailto:g.simioli@live.it}{g.simioli@live.it}}
    %\skill{\faGlobe~ Website}{\href{https://dmf.unicatt.it/miti/}{https://dmf.unicatt.it/miti/}}
    %\skill{\faGithub~ GitHub}{\href{https://github.com/MasterToninus}{MasterToninus}}    
\end{minipage}
%\textit{PhD in mathematics with a focus  on  differential geometry (multisymplectic) and its applications to physics.}
\vspace{-0.25em}



%Splash Page
\begin{minipage}[t]{0.6\columnwidth}

	\block{Posizione Attuale}

    	\voice{Postdoctoral fellow (CIVIS-3i)}
			{02/2024 -- Now}
			{Sapienza Universit\'a di Roma, Italy;}
			{https://civis3i.univ-amu.fr/en/civis3i-alliance-programme}       
			{Development of advanced mathematical models and scientific computing tools within a large European research consortium. Strong focus on interdisciplinary collaboration and project coordination.}

    \block{Esperienze Professionali}

    \voice{Fisioterapista (Libera professione)}
    {12/2017 -- 09/2018}
    {Istituto geriatrico Milanese (IGM), Milano (MI)}
    {}
    {Responsabile di un intero piano di degenza geriatrica (57 pazienti circa). dovendomi dividere tra la riabilitazione, le loro cartelle e tutta la burocrazia annessa.}
    \voice{Fisioterapista (Libera professione)}
    {06/2014 -– 12/2017}
    {Centro I.R.M.I. (Istituto riabilitativo del mezzogiorno d'Italia), Giugliano in Campania (NA)}
    {}
    {Fisioterapista e terapista occupazionale per pazienti in convitto ex articolo 26.}
    \voice{Fisioterapista (Collaborazione occasionale)}
    {07/2014 -- 08/2014}
    {Clinic Center, Fuorigrotta (NA)}
    {}
    {In questo periodo ho lavorato come sotituzione estiva in questa clinica di riabilitazione,
avendo la possibilità di lavorare con i pazienti in piena libertà e in qualsiasi campo riabilitativo.
Questa per  me è stata la prima esperienza in una vera clinica riabilitativa.}
    %
    \voice{Fisioterapista (Collaborazione occasionale)}
    {02/2014 -- 04/2015}
    {Centro Golia , Aversa (CE)}
    {}
    {Il centro è convenzionato con l'Asl e si occupa di tutte le patologie ortopediche.
    In questo periodo ho avuto la possibilità di acquisire ulteriore esperienza nell'ambito ortopedico e con i  mezzi fisici utilizzati in riabilitazione: Tecar, magneto, tutti i tipi di correnti, laser, ultrasuoni.
    Le patologie trattate sono tutte quelle di tipo ortopedico.}
    %
    \voice{Fisioterapista}
    {10/2013 -- 11/2013}
    {Centro Aequilibrium , Caserta (CE)}
    {}
    {In questo periodo ho fatto un po' d'esperienza all'interno di questo studio di riabilitazione diretto dal dott. Bruno. Le problematiche trattate erano diverse anche se quasi tutte incentrate nell'ambito ortopedico e posturale:  gonartrosi, coxartrosi, ernie, lesioni muscolarie  e tendinee, scoliosi ecc...
Per quanto riguarda i mezzi fisici, ho utilizzato : Tecarterapia, correnti Tens, magnetoterapia, laser e ultrasuoni.\vspace{-0.25cm}}
 

    \block{Formazione Accademica}

    
\voice{Laurea in Fisioterapia – Fisioterapista}
{02/11/2008 – 23/04/2012}
{Università degli Studi di Napoli Federico II, Facoltà di Medicina e Chirurgia}
{}
{
Tirocinio per la tesi finale presso la \textbf{Fondazione Istituto Antoniano}, Ercolano (Napoli).\\
Esperienza diretta con pazienti affetti da autismo, sindrome di Down e ritardo mentale.\\
\textbf{Tesi:} “Autismo e dorso curvo posturale. Un percorso riabilitativo tra attività fisica e musicoterapia”.\\
Votazione: \textbf{107/110}.\\
Iscrizione al corso di laurea in Fisioterapia (a.a. 2008/2009) presso il Policlinico Universitario Federico II.
}
    \voice{Bachelor degree in Physics}
        {October 2010}
        {Universit\'a degli Studi di Milano - Bicocca}
        {https://www.fisica.unimib.it/en}       
        {Theoretical Physics curriculum with emphases on: Mathematical-Physics, Classical and Quantum Mechanics.\vspace{-0.25cm}}

















\end{minipage} 
\hfill
\begin{minipage}[t]{0.375\columnwidth}
   
    \block{Bio}

	\skill{Nascita}{1987, Mugnano di Napoli}
	\skill{Cittadinanza}{Italiana}
	\skill{Stato Civile}{Sposata con 1 figlia}



    \block{Lingue} 
        
    \skill{Italian}{Madrelingua}
    \skill{Francese}{Professionale Completo}
    \skill{Inglese}{Professionale Intermedio}
        

   \block{Competenze Riabilitative} 
    
    \skillbar{Pacchetto Office}
    {1cm}
    {1cm}
    {0.25cm}




    \block{Competenze Informatiche} 
      
    \skillbar{Pacchetto Office}
    {1cm}
    {1cm}
    {0.25cm}


    \block{Risultati Professionali}

    \skill{Anni di esperienza}{8 }
    \skill{Corsi di formazione}{6 (50+ hours) }
    \skill{Boh??}{20+ }

    
\block{Albo Professionali}

\voice{French Habilitation to Assistant Professor }
  {02/2022}
  {Minist\'ere français de l'Enseignement sup\'erieur et de la Recherche}
  {https://www.galaxie.enseignementsup-recherche.gouv.fr/ensup/qualification/Resultats_2022/qualifies_MCF_2022.pdf}
  {National qualification in pure mathematics for university-level teaching and research positions. (Maitre de conf\'erences, Section 25)}
\voice{Albo Italiano Abroad Grant}
  {05/2023}
  {Istituto Nazionale di Alta Matematica}
  {https://www.altamatematica.it/wp-content/uploads/2022/12/bando-estero-2022-2023-1.pdf}
  {Sospeso perchè non riconosceva i corsi di formazione fatti in Francia.}




\end{minipage}


\clearpage

    \block{Corsi di Formazione}

    \voice{Drenaggio Autogeno (tecnica Chevalier) con Hugues Gauchez}
    {05/2024 – 09/2024}
    {P2R-Formation, Lyon, Francia}
    {}
    {
    Corso intensivo su tre weekend sul \textbf{drenaggio autogeno} nella gestione delle patologie respiratorie ostruttive.\\
    Tecnica validata dalla \emph{Airways Clearance Technique Review (IPGCF)} per fibrosi cistica e asma.\\
    Approccio centrato sulla consapevolezza respiratoria e sull’aderenza terapeutica del paziente.
    }
    %
    %
    \voice{Diploma Universitario in Fisioterapia Respiratoria e Cardiovascolare}
    {09/2022 – 05/2023}
    {Université Claude Bernard Lyon 1, Francia}
    {}
    {
    Percorso annuale di riabilitazione cardiorespiratoria basato su evidenze cliniche.\\
    \textbf{Corsi collegati:}\\
    06/12/21 – 09/12/21: Fisioterapia respiratoria in rianimazione e ventilazione invasiva/non invasiva – Alister, Mulhouse.\\
    01/12/20 – 03/12/20: Corso pratico di fisioterapia respiratoria in rianimazione – Alister, Mulhouse.\\
    Temi principali: diagnosi fisioterapica, gestione della ventilazione meccanica, decondizionamento e ossigenoterapia.
    }
    %
    %
    \voice{Metodo Ticchi e Trigger Point}
    {10/2016 -- 12/2016}
    {Istituto, luogo???}
    {}
    {
    Formazione sul riconoscimento e trattamento delle zone di iper-irritabilità muscolare.\\
    Approccio clinico basato su casi reali e valutazione funzionale mirata.\\
    Sviluppo di competenze analitiche e relazionali nel contesto terapeutico.
    }
    %
    %
    \voice{Corso di Ginnastica Ortopedica Morfologica Mézières (GOMM)}
    {02/2016}
    {Scuola di Formazione Jean-Marc Cittone}
    {}
    {
    Formazione sanitaria-riabilitativa basata sul metodo Mézières.\\
    Applicazioni per il mantenimento posturale e il benessere fisico personale.
    }
    %
    %
    \voice{Master Mézières Plus}
    {01/2016}
    {Scuola di Formazione Jean-Marc Cittone}
    {}
    {
    Corso avanzato di perfezionamento (5° stage).\\
    Approfondimenti su scoliosi e problematiche cervicali, con focus su tecniche complementari.
    }
    %
    %
    \voice{Corso Mézières Plus}
    {05/2015 –- 11/2015}
    {Scuola di Formazione Jean-Marc Cittone}
    {}
    {
    Apprendimento delle tecniche per il riequilibrio morfologico (bacino, spalle, mandibola).\\
    Applicazioni in traumatologia, ortopedia, neurologia e medicina sportiva.
    }
    %
    %

    \longvoice{Introduction to Scientific Programming using GPGPU and CUDA}
        {Training Course}
        {https://web.archive.org/web/20170623171128/http://www.hpc.cineca.it/content/introduction-to-gpu}
        {March 2014}
        {CINECA-SCAI}
        {Milan}
        {Italy}
        {}
    









\end{document}