%=======================================================================%
%      CV - non accademico
%=======================================================================%
%  --- NON GENERATO AUTOMATICAMENTE ---
%
%   non ha senso automatizzarlo per adesso, le cose da aggiungere sono poche per ora.
%
%
%
%=======================================================================%





\documentclass[a4paper]{article}


\usepackage[utf8]{inputenc}
\usepackage[T1]{fontenc}
\usepackage{fontawesome}
\usepackage[margin=1.5cm]{geometry}
% \addtolength{\topmargin}{-0.5cm}
\usepackage{xcolor}
\usepackage{hyperref}
\usepackage{fancyhdr}
\usepackage{lastpage}
\pagestyle{fancy}
\fancyhf{}
\rhead{Giuseppa Simioli}
\lhead{Curriculum Vitae}
\lfoot{\footnotesize\href{https://github.com/MasterToninus/cv}{Generated on \today} }
\rfoot{\thepage \hspace{1pt} / \pageref{LastPage}}

%\usepackage[backend=bibtex, style=nature]{biblatex}
%\addbibresource{../data/publications.bib}

\setlength\parindent{0pt}


%barcharts
\newcommand{\crule}[3][black]{\textcolor{#1}{\rule{#2}{#3}}}
\newcommand{\progbar}[4][black]{\crule[#1]{#3}{#4}\crule[#1!10!white!90]{#2}{#4}}

%titles
\newcommand{\block}[1]{\hrule \vspace{0.2cm} \textbf{\Large #1} \vspace{0.2cm}}
\newcommand{\Block}[1]{\hrule \vspace{0.2cm} \textbf{\huge #1} \vspace{0.2cm}}
\newcommand{\blockNp}[1]{\newpage \textbf{\Large #1} \vspace{0.2cm}}

%longvoice
%expected: Event_Title,Type,Url, Approx_Date, Organization, Location, Country
\newcommand{\longvoice}[8]{
    \begin{tabular}{p{0.83\linewidth} p{0.17\linewidth} }
        \textbf{\href{#3}{#2: #1}} & #4 
        \\ 
        \textit{#5, #6, #7;} & {\small\emph{#8}}
    \end{tabular}
    \vspace{.5em}
    }
%voice
%expected: what, when, where, url, info
\newcommand{\voice}[5]{\href{#4}{\textbf{#1}} \hfill #2 \\ \textit{#3} \\ {\small #5} \vspace{0.2cm} \\}
\newcommand{\skill}[2]{\textbf{#1} \hfill #2 \\}
\newcommand{\skillbar}[4]{\textbf{#1} \hfill \progbar{#2}{#3}{#4} \\}
\newcommand{\txt}[1]{#1\\}




%---------- Document -----------
\begin{document}


%Header
\begin{minipage}[c]{0.6\columnwidth}
    \center{\huge \textbf{Giuseppa SIMIOLI}}
    \\[-1.5em]
    \textit{\flushleft{\normalsize \textit{Fisioterapista con oltre dieci anni di esperienza in ambito pubblico e privato, specializzata in riabilitazione respiratoria, oncologica e posturale secondo il metodo Mézières.}}
    \\
    }    
    \end{minipage} 
\hfill
\begin{minipage}[t]{0.375\columnwidth}
    \vspace{-1.5em}
    \skill{\faPhone~ Phone (ita)}{+39~}
    \skill{\faPhone~ Phone (fra)}{+32~}
    \skill{\faEnvelope~ Mail}{\href{mailto:g.simioli@live.it}{g.simioli@live.it}}
    %\skill{\faGlobe~ Linkedin}{\href{https://www.linkedin.com/in/giuseppa-simioli-45777544/}{giuseppa-simioli-45777544}}
    %\skill{\faGithub~ GitHub}{\href{https://github.com/MasterToninus}{MasterToninus}}    
\end{minipage}
%\textit{PhD in mathematics with a focus  on  differential geometry (multisymplectic) and its applications to physics.}
\vspace{1.25em}



%Splash Page
\begin{minipage}[t]{0.6\columnwidth}

	\block{Posizione Attuale}

    \voice{Fisioterapista (Dipendente TI)}
    {01/2021 -- Oggi}
    {Centre Léon Bérard, Lyon, Francia}
    {https://www.centreleonberard.fr/en}
    {
        Ospedale oncologico privato di interesse nazionale.
    Attività clinica nei reparti di chirurgia viscerale e ortopedica, terapia palliativa, chirurgia ORL e toracica, ed ematologia.
    Responsabilità dirette nel percorso di riabilitazione post-operatoria e nel supporto respiratorio e funzionale dei pazienti oncologici.%\\
    %Esperienza altamente formativa sia dal punto di vista clinico che umano, maturata in un contesto multidisciplinare e ad alta intensità assistenziale.
    }

    \block{Esperienze Professionali}

    \voice{Fisioterapista (Dipendente TI)}
    {09/2019 -- 01/2021}
    {Centre Hospitalier de Valenciennes, Hauts-de-France, Francia}
    {https://www.ch-valenciennes.fr/}
    {
    Riabilitazione respiratoria in contesto ospedaliero durante il periodo di emergenza Covid-19. Ruolo operativo nelle funzioni cliniche quotidiane con pazienti affetti da patologie respiratorie acute.%\\
    %Esperienza di forte crescita clinica e organizzativa in un ambiente multidisciplinare ad alta intensità assistenziale.
    }
    %
    %
    \voice{Fisioterapista (Libera professione)}
    {12/2017 -- 09/2018}
    {Istituto geriatrico Milanese (IGM), Milano (MI)}
    {https://www.igm-care.it/service/istituto-geriatrico-milanese/}
    {Responsabile di un intero piano di degenza geriatrica (57 pazienti circa). dovendomi dividere tra la riabilitazione, le loro cartelle e tutta la burocrazia annessa.}
    %
    \voice{Fisioterapista (Libera professione)}
    {06/2014 -– 12/2017}
    {Centro I.R.M.I. (Istituto riabilitativo del mezzogiorno d'Italia), Giugliano in Campania (NA)}
    {https://www.aiopcampania.it/mostrascheda.asp?id=1425}
    {Fisioterapista e terapista occupazionale per pazienti in convitto ex articolo 26.}
    %
    \voice{Fisioterapista (Collaborazione occasionale)}
    {07/2014 -- 08/2014}
    {Clinic Center, Fuorigrotta (NA)}
    {https://cliniccenter.eu/}
    {Sostituzione estiva in clinica riabilitativa, trattamento indipendente in qualsiasi ambito riabilitativo.
    %avendo la possibilità di lavorare con i pazienti in piena libertà e in qualsiasi campo riabilitativo.
    }
    %
    \voice{Fisioterapista (Collaborazione occasionale)}
    {02/2014 -- 04/2015}
    {Centro Golia , Aversa (CE)}
    {https://centrogolia.com/}
    {
        Riabilitazione ortopedica in centro convenzionato ASL.
        Trattamenti con mezzi fisici: Tecar, magneto, tutti i tipi di correnti, laser, ultrasuoni.}
    %
    \voice{Fisioterapista (Collaborazione occasionale)}
    {10/2013 -- 11/2013}
    {Centro Aequilibrium , Caserta (CE)}
    {https://www.fisioterapistacaserta.it/About}
    {Riabilitazione nell'ambito ortopedico e posturale:  gonartrosi, coxartrosi, ernie, lesioni muscolarie  e tendinee, scoliosi ecc...
    Trattamenti con mezzi fisici: Tecarterapia, correnti Tens, magnetoterapia, laser e ultrasuoni.}
 %
%
    \voice{Masso-fisioterapista}
    {02/2013 -- 06/2013}
    {Angelo Caroli – Wellness and Beauty Spa, Milano (MI)}
    {https://www.angelocaroli.com/}
    {
    Attività in contesto wellness, tecniche di massaggio rilassante, anticellulite e terapeutico.
    Trattamento di contratture e dolori del rachide mediante manovre specifiche e allungamenti muscolari mirati.
    \vspace{-0.25cm}}




















\end{minipage} 
\hfill
\begin{minipage}[t]{0.375\columnwidth}
   
    \block{Bio}

	\skill{Nascita}{1987, Mugnano di Napoli}
	\skill{Cittadinanza}{Italiana}
	\skill{Stato Civile}{Sposata con 1 figlia}



    \block{Lingue} 
        
    \skill{Italiano}{Madrelingua}
    \skill{Francese}{Professionale Completo}
    \skill{Inglese}{Professionale Intermedio}
        

   \block{Competenze Riabilitative} 
    
    \skillbar{Ortopedico}
    {1cm}
    {1cm}
    {0.25cm}
    \skillbar{Posturale}
    {1cm}
    {1cm}
    {0.25cm}
    \skillbar{Respiratorio}
    {1cm}
    {1cm}
    {0.25cm}
    \skillbar{Oncologico}
    {1cm}
    {1cm}
    {0.25cm}
    \skillbar{Geriatrico}    
    {1cm}
    {1cm}
    {0.25cm}
    \skillbar{Neurologico}
    {1cm}
    {1cm}
    {0.25cm}
    \skillbar{Sportivo}
    {1cm}
    {1cm}
    {0.25cm}
    \skillbar{Pediatrico}
    {1cm}
    {1cm}
    {0.25cm}             
    \skillbar{Occupazionale}
    {1cm}
    {1cm}
    {0.25cm}
    \skillbar{Cardiologico}
    {1cm}
    {1cm}
    {0.25cm}

    \skill{Anni di esperienza}{11 }
    \skill{Corsi di formazione}{6 (50+ ore) }
    \skill{Boh??}{20+ }


    \block{Altre Competenze } 
      
    \skillbar{Office}
    {1cm}
    {1cm}
    {0.25cm}
    \skillbar{Lavoro individuale}
    {1cm}
    {1cm}
    {0.25cm}
    \skillbar{Lavoro in equipe}
    {1cm}
    {1cm}
    {0.25cm}  
    \skillbar{Organizzazione equipe}
    {1cm}
    {1cm}
    {0.25cm}
    \skillbar{Comunicazione}
    {1cm}
    {1cm}
    {0.25cm}
    \skill{Corsi di formazione}{9 (50+ ore) }
    \skill{Patente ECDL}{??}
    % Patente Europea del Computer (ECDL)	
    %Istituto Gama
    %Sistema operativo Windows XP: Word, Exel, Access, Powerpoint, Reti informatiche




    
\block{Albi Professionali}

\voice{Albo Francese }
  {11/2020 -- Oggi}
  {Ordre des masseurs-kinésithérapeutes}
  {https://www.ordremk.fr/}
  {...}
\voice{Albo Italiano}
  {05/2012 -- 05/2021}
  {FNOFI}
  {https://www.fnofi.it/}
  {Sospeso perchè non riconosceva i corsi di formazione fatti in Francia.}




\end{minipage}


%\clearpage

    \block{Formazione}

    \voice{Laurea in Fisioterapia}
    {11/2008 –- 04/2012}
    {Università degli Studi di Napoli Federico II%, Facoltà di Medicina e Chirurgia
    }
    {https://m78.corsidistudio.unina.it/}
    {
    Corsi e tirocini presso il \href{https://www.policlinico.unina.it/flex/cm/pages/ServeBLOB.php/L/IT/IDPagina/1}{Policlinico Universitario Federico II}.
    Tirocinio Finale presso la \href{https://www.istitutoantoniano.it/}{Fondazione Istituto Antoniano} di Ercolano (NA) con esperienza diretta con pazienti affetti da autismo, sindrome di Down e ritardo mentale.\\
    \textbf{Tesi:} “Autismo e dorso curvo posturale. Un percorso riabilitativo tra attività fisica e musicoterapia”, relatore ???, Votazione: \textbf{107/110}.
    \vspace{-0.25cm}
    }

    \block{Formazione Post-Laurea}

    \voice{Drenaggio Autogeno (tecnica Chevalier) con Hugues Gauchez}
    {05/2024 – 09/2024}
    {P2R-Formation, Lyon, Francia}
    {}
    {
    Corso intensivo su tre weekend sul \textbf{drenaggio autogeno} nella gestione delle patologie respiratorie ostruttive.\\
    Tecnica validata dalla \emph{Airways Clearance Technique Review (IPGCF)} per fibrosi cistica e asma.\\
    Approccio centrato sulla consapevolezza respiratoria e sull’aderenza terapeutica del paziente.
    %Formazione di tre lunghi weekend sul Drenaggio Autogeno, che  ha dimostrato la sua efficacia nella gestione delle malattie respiratorie ostruttive (grado B). Il drenaggio autogeno (DA) è stato avviato nell'asma e validato nella fibrosi cistica dalla Airways Clearance Technique Review dell'IPGCF (International Physiotherapy for Cystic Fibrosis). Erroneamente pubblicizzato come drenaggio autonomo e come tecnica di espirazione delicata e lenta, in realtà consente al paziente e al fisioterapista di connettersi in modo che ogni ciclo respiratorio fornisca benessere ventilatorio. Poiché le patologie respiratorie sono spesso croniche, l'aderenza del paziente al trattamento dipende dai benefici che ne trae.
    }
    %
    %
\voice{Diploma Universitario in Fisioterapia Respiratoria e Cardiovascolare}
{09/2022 -- 05/2023}
{Université Claude Bernard Lyon 1, Francia}
{}
{
Percorso universitario annuale in riabilitazione cardiorespiratoria.\\
Approfondimento teorico-pratico sulle più recenti metodologie di fisioterapia respiratoria e cardiovascolare.\\
Esperienza centrata sull’autovalutazione clinica e sull’approccio evidence-based per il trattamento dei pazienti.
}
%
%
\voice{Fisioterapia respiratoria in rianimazione e ventilazione meccanica (invasiva e non invasiva)}
{12/2021}
{Alister, Mulhouse, Francia}
{}
{
Formazione teorico-pratica sulla fisioterapia respiratoria in terapia intensiva.\\
Trattazione di diagnosi fisioterapica, valutazione respiratoria e gestione dei pazienti con patologie acute e croniche.\\
Approfondimento su ventilazione meccanica, ossigenoterapia e collaborazione interdisciplinare in rianimazione.
}
%
%
\voice{Corso pratico di fisioterapia respiratoria in rianimazione}
{12/2020}
{Alister, Mulhouse, Francia}
{}
{
Corso intensivo con esercitazioni teoriche e applicazioni pratiche in gruppo.\\
Approfondimento sulle tecniche di decongestione manuale e strumentale, ventilazione e tosse assistita.\\
Trattazione di valutazione funzionale, gestione del paziente in ventilazione e ruolo del fisioterapista nel team multidisciplinare.
}
%
%
\voice{Corso di I livello ETC (Esercizio Terapeutico Conoscitivo e Confronto tra Azioni)}
{05/2018 -- 06/2018}
{dove??}
{}
{
Corso avanzato sul metodo \textbf{Esercizio Terapeutico Conoscitivo (ETC)}.\\
Formazione mirata a sviluppare autonomia nella progettazione del trattamento riabilitativo attraverso il confronto tra azioni.\\
Approfondimento sull’osservazione e interpretazione del comportamento del paziente neuroleso, sulla strutturazione del ragionamento riabilitativo e sull’adattamento degli strumenti neurocognitivi alle diverse fasi del recupero motorio.\\
Corso articolato in una settimana di teoria e due di pratica.
}
%
%
\voice{Corso sul Metodo Ticchi e Trigger Point}
{10/2016 -- 12/2016}
{dove??}
{}
{
Formazione sul riconoscimento e trattamento delle zone di iper-irritabilità muscolare, alla base dei dolori muscolo-scheletrici.\\
Apprendimento delle tecniche di valutazione e trattamento funzionale guidate dal Prof.\ Ticchi.\\
Acquisizione di competenze operative e capacità di analisi del contesto terapeutico attraverso casi clinici reali.
}
%
%
\voice{Corso di Ginnastica Ortopedica Morfologica Mézières (GOMM)}
{02/2016}
{Scuola di Formazione Jean-Marc Cittone}
{}
{
Corso professionale basato sui fondamenti del metodo \textbf{Mézières}.\\
Formazione sanitaria-riabilitativa orientata al consolidamento posturale e al benessere fisico personale.
}
%
%
\voice{Master Mézières Plus (5° stage di perfezionamento)}
{01/2016}
{Scuola di Formazione Jean-Marc Cittone}
{}
{
Corso avanzato sul metodo \textbf{Mézières}, focalizzato su tecniche specifiche per il trattamento delle disfunzioni cervicali e posturali.\\
Approfondimento su scoliosi e patologie correlate, finalizzato alla piena padronanza del metodo.
}
%
%
\voice{Corso Mézières Plus}
{05/2015 -- 11/2015}
{Scuola di Formazione Jean-Marc Cittone}
{}
{
Formazione sulle tecniche di riequilibrio morfologico e correzione delle asimmetrie strutturali (bacino, spalle, mandibola).\\
Applicazioni pratiche in traumatologia, ortopedia, neurologia e medicina sportiva.
}


    


    \block{Volontariato}

\voice{Volontaria presso l’associazione Tribunale del Malato}
{09/2005 -- 05/2006}
{Ospedale San Paolo, Napoli}
{}
{
Attività di supporto ai pazienti e collaborazione con il personale sanitario nell’ambito dell’assistenza ospedaliera.\\
Esperienza formativa nel settore sanitario, con attenzione ai diritti del paziente e alla comunicazione medico–paziente.
}






\end{document}